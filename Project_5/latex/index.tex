This program contains the necessary functions to implement
\begin{DoxyItemize}
\item Setup() that takes the inputs from the terminal and sets up the board according to the given specifications.
\item move\+Foward() takes in a car number and moves the car Forward if it is possible.
\item move\+Backward() takes in a car number and moves the car Backward if it is possible.
\item is\+Solved() to check if car 0 is at the end of the board.
\item Print() this was used for my own testing purposes to print out the board and its data.
\item clear() used to clear the board , essentially the same as setup but used to make sure that the board is reset so the next board can be setup.
\item \hyperlink{_rush_hour_8cpp_a03450682e7fd05eadf4f39c3a2b105ad}{Solve\+It()} which will be used to try to solve the board if possible.
\item board\+To\+String() which will convert our board into a string.
\end{DoxyItemize}

We implemented it with two classes, called \hyperlink{class_board}{Board} and Car. The board contains all of the information about the board including the number of cars on the board and the number of current moves taken to solve the board. To prevent seg faults, We had to keep in mind the boundaries of the board when moving the cars inside the board. For my implementation, I created an int board, where 0 marked empty spaces with no cars that other cars could freely move to and (i+1) the car number in which it was entered in the terminal( so the first car will be 1, second car will be 2 and so on) on the board represented where the space was occupied by a car and another car could not move there. I used the board to see and decide if the cars could move forward or backward based on its current position. 